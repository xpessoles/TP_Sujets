\documentclass[10pt,fleqn]{article} % Default font size and left-justified equations
\usepackage[%
    pdftitle={Hyperstatisme},
    pdfauthor={Xavier Pessoles}]{hyperref}

    
\input{style/new_style}
\input{style/macros_SII}
\usepackage{multicol}
\usepackage{siunitx}
%\usepackage{picins}
\fichetrue
%\fichefalse

\proftrue
\proffalse

\tdtrue
%\tdfalse

\courstrue
\coursfalse


\def\classe{\textsf{PSI$\star$}}
\def\xxnumpartie{Cycle xx}
\def\xxpartie{Préparation aux épreuves orales}

\def\xxnumchapitre{Filière PSI \vspace{.2cm}}
\def\xxchapitre{\hspace{.12cm} }


\def\discipline{Sciences \\Industrielles de \\ l'Ingénieur}
\def\xxtete{Sciences Industrielles de l'Ingénieur}


\def\xxactivite{TD}
\def\xxauteur{\textsl{Xavier Pessoles}}


\def\xxtitreexo{Robot MaxPID}
\def\xxsourceexo{\hspace{.2cm} \footnotesize{Laboratoire de PSI}}


  
\def\xxposongletx{2}
\def\xxposonglettext{1.45}
\def\xxposonglety{20}
%\def\xxonglet{Part. 1 -- Ch. 3}
\def\xxonglet{\textsf{Cycle xx}}

\def\xxactivite{TD}
\def\xxauteur{\textsl{Xavier Pessoles}}

\def\xxcompetences{%
\vspace{-.5cm}
\footnotesize{
\textsl{%
%\textbf{Savoirs et compétences :}\\
%\vspace{-.2cm}
%\begin{itemize}[label=\ding{112},font=\color{ocre}] 
%\item \textit{Mod2.C34} : chaînes de solides;
%\item \textit{Mod2.C34} : degré de mobilité du modèle;
%\item \textit{Mod2.C34} : degré d’hyperstatisme du modèle;
%\item \textit{Mod2.C34.SF1} : déterminer les conditions géométriques associées à l’hyperstatisme;
%\item \textit{Mod2.C34} : résoudre le système associé à la fermeture cinématique et en déduire le degré de mobilité et d’hyperstatisme.
%\end{itemize}
}}}


\def\xxfigures{
%\includegraphics[width=.8\textwidth]{images/fig_01}
}%figues de la page de garde


\def\xxpied{%
Préparation aux oraux\\%dans le but de déterminer les contraintes géométriques dans les mécanismes\\% afin de valider leurs performances.\\
%Chapitre 2 -- \xxactivite%
}

\setcounter{secnumdepth}{5}
%---------------------------------------------------------------------------


\begin{document}
%\chapterimage{png/Fond_Cin}
\input{style/new_pagegarde}
\vspace{4.5cm}
\pagestyle{fancy}
\thispagestyle{plain}


\def\columnseprulecolor{\color{ocre}}
\setlength{\columnseprule}{0.4pt} 

\begin{multicols}{2}
\subsection*{Maxpid}
\setcounter{exo}{0}

Vous commencerez par ouvrir un document PowerPoint ou Word ou autre dans lesquels vous sauvegarderez ou synthétiserez vos résultats.

\textbf{Ouvrir le modèle Solidworks Maxpid.SLDASM.}

\subparagraph{}\textit{Réaliser le graphe des liaisons associé au mécanisme en précisant l'ensemble des actions mécaniques s'exerçant sur le système dans le modèle Méca 3D.} 

\subparagraph{}\textit{Initier un calcul mécanique et justifier la page \textbf{Analyse du mécanisme} : 
\begin{itemize}
\item nombre de cycles indépendants;
\item nombre d'équations et d'inconnues cinématiques;
\item nombre de pièces;
\item nombre d'équations statiques et d'inconnues statiques;
\item mobilité et hyperstatisme.
\end{itemize}}


\subsection*{Étude géométrique}
\subparagraph{}\textit{Déterminer le nombre de tours réalisés par la vis pour un quart de tour du bras.}

\subparagraph{}\textit{Tracer la loi entrée-sortie géométrique (position angulaire du bras en fonction de la position angulaire de la vis).}

\subparagraph{}\textit{Estimer (après linéarisation) le rapport de réduction du système de transmission dans 3 zones de fonctionnement du MaxPID. }



\subsection*{Étude cinématique}

On souhaite que le bras se déplace d'un quart de tour en 1 seconde (en faisant ici l'hypothèse d'une vitesse constante), en faisant l'hypothèse que la vitesse est constante pendant tout le mouvement. 

\subparagraph{}\textit{Tracer la vitesse de vis par rapport au stator moteur en fonction du temps. Quelle sera la vitesse maximale de la vis.}

\subparagraph{}\textit{En imposant sa vitesse maximale à la vis, quelle sera la vitesse maximale du bras ?}

\subparagraph{}\textit{Déterminer alors combien de temps le bras mettra pour faire un quart de tour ?}


\subsection*{Étude statique}

\subparagraph{}\textit{Quel couple doit fournir le moteur pour réaliser un quart de tour de bras quand le Maxpid est à plat (dans le plan horizontal) ?}

\subparagraph{}\textit{Quel couple doit fournir le moteur pour réaliser un quart de tour de bras quand le Maxpid est vertical (dans le plan vertical) ? Expliquer.}

\subparagraph{}\textit{Donner l'influence du nombre de masses en bout de bras sur le couple moteur pour un quart de tour du bras.}


\subparagraph{}\textit{Donner le couple maximal à fournir en statique, lorsque le MaxPID est en positionnement vertical.}

\subsection*{Étude énergétique}

\subparagraph{}\textit{Déterminer pour quel ensemble l'énergie cinétique est prépondérante lorsque le bras fait un quart de tour.}


\subsection*{Étude dynamique}


\subparagraph{}\textit{Quel couple doit fournir le moteur pour réaliser un quart de tour de bras quand le Maxpid est à plat ?}

\subparagraph{}\textit{Quel couple doit fournir le moteur pour réaliser un quart de tour de bras quand le Maxpid est vertical ? Expliquer.}

\subparagraph{}\textit{Donner l'influence du nombre de masses en bout de bras sur le couple moteur pour un quart de tour du bras.}


\subparagraph{}\textit{Donner le couple maximal à fournir en dynamique, lorsque le MaxPID est en positionnement vertical et que la loi de mouvement du bras est un trapèze de vitesse.}

\end{multicols}



\end{document}

\subparagraph{}\textit{}
\ifprof
\begin{corrige}~\\
\end{corrige}
\else
\fi

\begin{center}
\includegraphics[width=\linewidth]{images/img_04}
%\textit{}


\end{center}

